%%%%%%%%%%%%%%%%%%%%%%%%%%%%%%%%%%%%%%%%%%%%%%%%%%
%% Bachelor's & Master's Thesis Template        %%
%% Copyleft by Dawid Weiss & Marta Szachniuk    %%
%% Faculty of Computing and Telecommunication   %%
%% Poznan University of Technology, 2020        %%
%%%%%%%%%%%%%%%%%%%%%%%%%%%%%%%%%%%%%%%%%%%%%%%%%%


% Szkielet dla pracy licencjackiej pisanej w języku polskim.

\documentclass[english,bachelor,a4paper,oneside]{ppfcmthesis}


\usepackage[utf8]{inputenc}
\usepackage[OT4]{fontenc}

%--------------------------------------
% Strona tytułowa
%--------------------------------------

% Autorzy pracy, jeśli jest ich więcej niż jeden
% wstaw między nimi separator \and
\author
{%
   Patryk Kościk \album{144635}
}
\authortitle{}                                % Do not change.

\title
{%
   Performance analysis of different emulation methods 
   in control systems and IoT devices on the example of
   the Renode Framework
}

% Your supervisor comes here.
\ppsupervisor{dr inż.~Adam Turkot} 

% Year of final submission (not graduation!)
\ppyear{2023}                                 


\begin{document}

% Front matter starts here
\frontmatter\pagestyle{empty}%
\maketitle\cleardoublepage%

%--------------------------------------
% Miejsce na kartę pracy dyplomowej
%--------------------------------------

\thispagestyle{empty}\vspace*{\fill}%
\begin{center}Tutaj będzie karta pracy dyplomowej;\\oryginał wstawiamy do wersji dla archiwum PP, w pozostałych kopiach wstawiamy ksero.\end{center}%
\vfill\cleardoublepage%

%--------------------------------------
% Spis treści
%--------------------------------------

\pagenumbering{Roman}\pagestyle{ppfcmthesis}%
\tableofcontents* 
\cleardoublepage % Zaczynamy od nieparzystej strony

%--------------------------------------
% Rozdziały
%--------------------------------------

%Najwygodniej jeśli każdy rozdział znajduje się w oddzielnym pliku
\mainmatter%

\chapter{Introduction}

In a embedded devices development context, platform simulation is referring to a method of simulating 
the behavior of an entire platform, or only a few selected components, in order to test and evaluate
the functionality and performance of the system before it is deployed. This can be done using specialized
software tools and techniques that simulate the hardware, software, and other components of the platform,
allowing developers to run and test their code in a virtual (simulated) environment before it is implemented
onto the physical hardware.

Platform simulation became an indispensable tool in the modern-day embedded system development. Such technology
provides each engineer with an dedicated, deterministic and reproducible system for development, testing and debugging.
This approach allows the developers to use advanced debugging software to perform a wide range of tests to ensure that
the embedded system is functioning correctly and meeting the desired performance and reliability requirements. 
This allows developers to detect, identify and fix bugs early in the development process, saving time and resources.

The advantages of the platform simulation also reach beyond the software development phase, and well into the further
phases of the product lifespan. One such advantage is the ability to easily integrate these systems into existing
continuous integration (CI) systems that check for regressions with each software/hardware iteration.
This is because the entire platform is contained within a simulation layer, making integration much simpler than it
would be with hardware platforms, which would require a hardware/software integration layer. An another example of
the benefit brought by emulation is an ability to create software, without having an hardware platform ready.
This is an especially important in the post COVID-19 pandemic times, where the supply chains have been massively
disrupted, with lead time on some parts, especially for the most advanced elements, rising as much as four times,
from about four weeks, in the begging of the year 2020, to over twenty weeks at the end of 2021
\cite{Covid19-AUTOMOTIVE} \cite{Covid19-LEAD-TIME} \cite{Covid19-LEAD-TIME-BLOOMBERG}. Hardware emulation allows for
the parallelization of the software and hardware development, enabling developers to work on both aspects
simultaneously and reducing the overall time and effort required to bring a product to market.

\section{Motivation and goals of the thesis}

A large part of the recognized platform and/or core architecture emulators use opcode translation to run guest code
on the host platform. The most prevalent and widely used open source tools in this category include Renode \cite{Renode}
and QEMU \cite{Qemu}.%
\footnote{The Renode's CPU translation library is partly based on QEMU}
Due to major optimizations (translation blocks, caching, block chaining, ...), this method of the simulation yields
great results performance-wise, but on the other hand such heavy efficiency improvements come with a price of greatly 
complicating the codebase, making it difficult to maintain and integrate new functionality. Another downside of the
translation based approach is the need to \textbf{execute} the guest code, this requires complicated just in
time (\textit{JIT}) recompilers. A possible alternative to translation based CPU simulators are the interpretation
based CPU simulators. This approach tend to be much simpler as these simulators do not require code recompilation,
and usually do not implement very complicated optimizations, instead focusing on simple, maintainable and easily
expandable implementation of the simulated core instructions. While the lack of optimizations results in simpler
code, the simulation performance might suffer.

The main goal of the thesis is to integrate an already existing interpretation-based CPU simulator within the Renode
embedded system simulation framework. Then to compare the performance and accuracy of the simulation using benchmarks
and real world applications such as automatic control systems, general purpose real-time operating systems 
\textit{(RTOS)}, and even higher complexity binaries, for example AI solutions for the embedded devices.
By doing so, we hope to gain a better understanding of the performance capabilities and limitations of this approach,
and to identify potential areas for improvement and future work in this topic.

% Not indenting here as this is connected to the previous paragraph
\noindent Intermediate goals of the thesis include:
\begin{itemize}
	\item{Description of the existing translation and interpretation based approach CPU simulators}
	\item{Integrating an chosen interpretation CPU simulator into the Renode framework}
	\item{Implementation of the test software on the simulated platform with data gathering}
	\item{Comparative analysis of interpreter-based and translation-based solutions with regard to the performance}
\end{itemize}

\section{Thesis organization}

The work is divided into the following chapters:
\begin{itemize}
	\item{Chapter 2 will provide an in-depth description of the Renode's translation CPU simulation library}.
	\item{Chapter 3 provides an overview of Dromajo, the chosen simulation library}.
	\item{Chapter 4 focuses on integrating Dromajo into the Renode framework}.
	\item{Chapter 5 describes the benchmarks and the test binaries and presents their results}.
	\item{Chapter 6 concludes the thesis and suggests future work}.
\end{itemize}

%--------------------------------------
% Literatura
%--------------------------------------

\bibliographystyle{unsrt}{\raggedright\sloppy\small\bibliography{bibliografia}}

%--------------------------------------
% Dodatki
%--------------------------------------

\cleardoublepage\appendix%
\newpage
% Removing all includes from this section breaks some conditional
\if

%--------------------------------------
% Informacja o prawach autorskich
%--------------------------------------

\ppcolophon

\end{document}
