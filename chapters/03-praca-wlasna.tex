
\chapter{Rozwinięcie}

Rozdziały dokumentujące pracę własną studenta: opisujące ideę, sposób lub metodę 
rozwiązania postawionego problemu oraz rozdziały opisujące techniczną stronę rozwiązania 
--- dokumentacja techniczna, przeprowadzone testy, badania i uzyskane wyniki. 

Praca musi zawierać elementy pracy własnej autora adekwatne do jego wiedzy praktycznej uzyskanej w
okresie studiów. Za pracę własną autora można uznać np.: stworzenie aplikacji informatycznej lub jej
fragmentu, zaproponowanie algorytmu rozwiązania problemu szczegółowego, przedstawienie projektu 
np.~systemu informatycznego lub sieci komputerowej, analizę i ocenę nowych technologii lub rozwiązań
informatycznych wykorzystywanych w przedsiębiorstwach, itp. 

Autor powinien zadbać o właściwą dokumentację pracy własnej obejmującą specyfikację założeń i 
sposób realizacji poszczególnych zadań
wraz z ich oceną i opisem napotkanych problemów. W przypadku prac o charakterze 
projektowo-implementacyjnym, ta część pracy jest zastępowana dokumentacją techniczną i użytkową systemu. 

W pracy \textbf{nie należy zamieszczać całego kodu źródłowego} opracowanych programów. Kod źródłowy napisanych
programów, wszelkie oprogramowanie wytworzone i wykorzystane w pracy, wyniki przeprowadzonych
eksperymentów powinny być umieszczone np. na płycie CD, stanowiącej dodatek do pracy.

\section*{Styl tekstu}

Należy\footnote{Uwagi o stylu pochodzą częściowo ze stron prof. Macieja Drozdowskiego~\cite{Drozdowski2006}.} 
stosować formę bezosobową, tj.~\emph{w pracy rozważono ......, 
w ramach pracy zaprojektowano ....}, a nie: \emph{w pracy rozważyłem, w ramach pracy zaprojektowałem}. 
Odwołania do wcześniejszych fragmentów tekstu powinny mieć następującą postać: ,,Jak wspomniano wcześniej, ....'', 
,,Jak wykazano powyżej ....''. Należy unikać długich zdań. 

Niedopuszczalne są zwroty używane w języku potocznym. W pracy należy używać terminologii informatycznej, która ma 
sprecyzowaną treść i znaczenie. 

Niedopuszczalne jest pisanie pracy metodą \emph{cut\&paste}, bo jest to plagiat i dowód intelektualnej indolencji autora.
Dane zagadnienie należy opisać własnymi słowami. Zawsze trzeba powołać się na zewnętrzne źródła. 

