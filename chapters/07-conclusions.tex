
\chapter{Summary and conclusions}

\section{Summary}

The thesis offers a comprehensive examination of various simulation methods employed in the field, ranging from
relatively simple API emulators to complex whole-system simulators. It delves into the fundamental limitations
encountered in the design of simulators, challenges faced by users and developers, and the crucial need for such
software in the context of automatic control, robotics, Internet of Things (IoT), and edge computing solutions.
Additionally, the thesis provides an overview of types of mechanisms utilized in central processing unit emulation
software and their intended purposes

The thesis specifically focuses on two fundamental approaches used in CPU simulators: the translation approach and the
interpretation approach. The translation approach is examined in-depth, with the analysis of mechanisms of code
translation using TCG back-end and front-end, providing a sample of translated code, and examination of the execution flow
in such simulators. Special attention is given to the optimization methods employed in these simulators, such as
translation block caching and chaining. This analysis concludes with an examination of the code flow, exception and
interrupt handling, and the important concept of helper functions. This approach is described using the example of QEMU
and Renode Translation Libraries, with accompanying code samples.

The interpretation approach is also extensively examined, with the work delving into the methods employed in such
simulators, providing an in-depth examination of the inner workings of the Dromajo simulator, and comparing the two
simulation techniques. The author goes on to implement the interpretation processor simulator (Dromajo) with Renode.
The work describes the necessary interfaces for CPU emulation, explains
the process of integrating an API into an existing C library, and details the steps involved in the integration of a new
processor emulator with the Renode framework. Finally, the thesis presents a performance analysis of both types of
simulators and conducts a comparison of the results.

\pagebreak

\section{Conclusions}

The performance analysis conducted in this work concludes that translation processor emulators exhibit a substantial
performance advantage in terms of execution speed. This conclusion holds true for the majority of payloads, with the
exception of less complex binaries, where the optimization methods employed in the translation approach can create a
non-negligible overhead.

However, this advantage in execution speed comes at the cost of increased memory usage and much higher complexity in
the codebase. These trade-offs must be carefully considered when choosing between translation and interpretation
simulators, as they might not only have an instantaneous impact on the performance of the project but might also
have long-lasting negative effects in terms of the complicated codebase, lower maintainability and reduced
expandability.
% this is quite negative, I must say...

Interpretation simulators usually also offer the additional advantage of being fully portable, they do not need
to be heavily customized to run on another platform. This is not the case with translation simulators, which must
implement an additional \textit{intermediate representation} to host the translation layer, and they need to maintain architecture-specific backends.

\section{Future work}

There are several areas where future work can be done to further improve the simulation methods discussed in this
thesis. Some possible directions for future research include:

\begin{itemize}
    \item{\textbf{Developing more advanced interpretation simulators}: The interpretation approach has the advantage of
     being fully portable, but it also has a lower performance compared to translation simulators. Research can be done
     to develop more optimization methods, that can improve performance while still maintaining the advantages of
     portability and ease of maintainability and expansion.}
     %
     \item{\textbf{Improving the portability of translation simulators}: Despite great performance, translation
     simulators suffer in terms of portability. Further work can be done to explore new approaches to this method, in
     order to improve the ease of maintainability, expandability, and overall developer friendliness.}
     %
     \item{\textbf{Real time simulation}: Many simulation applications require real-time performance, like control
     systems, robotics, and other dynamic systems. Further research can be done to explore new methods for achieving
     real-time performance in simulation, such as using multi-core processors, distributed computing, or specialized
     hardware.}
     %
     \item{\textbf{Further evaluation}: Additional tests can be performed against the given solutions to further increase understanding of their strengths and shortcomings.
     More detailed analysis of impact of specific optimisations of the translation-based simulator can also be performed, to gauge their relative benefit against maintenance cost.}
\end{itemize}
