\chapter{Performance analysis}

The following chapter contains the results of the performance analysis of Translation and Interpretation CPU simulators,
presents the results, and compares the performance of the different solutions. The first section of this chapter focuses
on the description of the test environment in terms of hardware, software, and simulated guest parameters and
characteristics. The second section describes the benchmarks and binaries and motivates their importance in the field
of automatics and robotics, especially in the context of the Industry 4.0. The last section presents and interprets
the results.

\section{Test enviroment}
In order to achieve representative and reproducible results, the test environment must be carefully controlled, as not
to introduce any inaccurate and misleading conclusions. All of the tests were performed on the following testbench:

\begin{table}[h!]
    \centering
    \begin{tabular}{l|l}
    CPU              & Intel(R) Core(TM) i7-8700             \\
    \hline
    Memory           & 32GB DDR4 @ 2133 Mhz                  \\
    \hline
    Operating System & Linux Fedora 36 (6.0.9-200)           \\
    \hline
    C stdlib         & GNU libc 2.35                         \\
    \hline
    .NET Runtime     & Mono JIT compiler version 6.12.0.122
    \end{tabular}
\end{table}

\noindent
In terms of the software, both Renode Translation Library and Dromajo are compiled using \texttt{-O3} optimization
level. In addition to that, the libraries are \textbf{not using} any architecture specific optimizations, as in,
they were compiled \textbf{without} \texttt{-march=native} and \texttt{-mtune=native} flags. This is important
to note, as using these flags might significantly improve the overall performance of the simulators, but since
this work only focuses on performance deltas, these flags do not provide any additional value, and in the worst
case might introduce unwanted behavior.

The simulated guest platform is a \textbf{BeagleV Starlight JH7100}, with the only difference between the benchmark test
cases being the used CPU emulator.

In order to provide deterministic results, every test case is performed with \textbf{SetGlobalSerialExecution}
flag set to true. This Renode setting forces emulated processors to perform sequentially, removing the variability
introduced by the host operating system scheduler, thus maintaining full determinism.

\pagebreak

\section{Benchmark payloads and tested parameters}

\begin{itemize}
    \item{\textbf{Zephyr RTOS boot-up routine} this test performs a Zephyr RTOS payload boot-up sequence and waits for
    a 'hello-world' messeage on a UART device \cite{ZephyrHello}.\\
    \textbf{Motivation:} This test will determine whether the performance benefits/detriments are noticeable when
    running a very simple payload. This is important as, the current trend in the industry heads into the
    direction of implementing decentralized in control, IoT and edge computing solutions, often using cheaper and
    smaller devices running leaner payloads.}
    %
    \item{\textbf{Zephyr RTOS PI calculating sample} This sample application calculates Pi independently in many
    threads, and demonstrates the benefit of multiple execution units (CPU cores) when compute-intensive tasks can be
    run in parallel, with no cross-dependencies or shared resources \cite{ZephyrPi}.\\
    \textbf{Motivation:} In the context of Industry 4.0, this payload might represent the arithmetic operations
    performed while calculating an advanced automatic control system, or other algorithmically and/or numerically
    advanced tasks.}
    %
    \item{\textbf{Zephyr RTOS TensorFlow Lite Micro Module} this test runs the TensorFlow Lite Micro Zephyr module,
    runs the payload, and waits for it to finish. The model included with the sample is trained to replicate a sine
    function and generates x values to print alongside the y values predicted by the model. The x values iterate from 0
    to an approximation of 2$\pi$ \cite{ZephyrTF}.\\
    \textbf{Motivation:} A lot of the Industry 4.0 endpoints, such as sensors and actuators, use simple AI/DNN
    deployments, either to prematurely filter collected data or to assist in the decision-making. The proposed sample
    is a good match, as the neural network is not overly complicated, and runs on a real-time operating system.}
    %
    \item{\textbf{Linux kernel boot-up} this test loads all necessary payloads to boot the Linux kernel.\\
    As the Linux kernel is a relatively complicated payload, this test checks a wide range of the CPU simulation use
    cases\\
    \textbf{Motivation:} In Industry 4.0 times, the Linux kernel had become an unrepeatable part of almost every
    automatic control, internet of things, robotics, etc. infrastructure stack. Because of that it is important to
    provide a qualitative and performance evaluation of the simulation.}
\end{itemize}

\noindent
Each test will be repeated three times, and evaluated under the following criteria:
\begin{itemize}
    \item{\textbf{Qualitative evaluation}, has the test finished successfully and returned correct data?\\
    Verified on case-by-case basics.}
    %
    \item{\textbf{Performance evaluation}, how long did the test take?\\To guarantee accuracy and validity, this metric is
     provided by Renode Virtual Time framework.}
\end{itemize}

% \section{Performance results}